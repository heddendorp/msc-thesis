\addchap{Abstract}

Software testing plays a critical role in ensuring system quality, reliability, and performance. 
However, the effectiveness of testing is limited by the presence of flaky tests, which are tests that can pass or fail randomly with the same version of the \acf*{sut} and configuration.
\acf*{e2e} tests in particular can be susceptible to flaky behavior, posing significant challenges to the software development process. 
While existing research has primarily focused on the detection of flaky unit tests, this thesis addresses the detection of flaky failures in \acs*{e2e} tests. 
We adapt a technique from \citeauthor*{bell_deflaker_2018} for detecting flaky Java unit test failures and apply it to \acs*{e2e} tests, which involve multiple programming languages and a different context for test execution. Our approach requires the instrumentation of code to collect coverage data during test execution. 
To evaluate our methodology, we selected two open source projects, \textsc{ArTEMiS} and \textsc{n8n}, which use the Cypress framework for \acs*{e2e} testing.

Our contributions include a novel methodology for detecting flaky failures in \acs*{e2e} testing, an evaluation of our approach on two open source projects, and guidelines for practitioners. 
Our evaluation shows that instrumentation can have a significant impact on test execution, and that our approach correctly identified flaky failures 42\% of the time. 
73\% of the failures identified by our approach were indeed flaky, and 27\% were false positives.
The results indicate that while the adapted approach has potential, it is not directly applicable to \acs*{e2e} tests, and further research is needed to improve its performance. 
This is due to the high implementation complexity of the approach, the impact of instrumentation on test execution, and the low recall of the approach.
Future work should also explore the applicability of our approach to different technology stacks, extend the coverage collection tool to support other test frameworks, and investigate the impact of instrumentation on test execution.