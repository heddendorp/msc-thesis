% !BIB TS-program = biber

% Add common preamble to the document
% !TeX root = ..\thesis.tex
% \documentclass[a4paper,12pt,twoside,listof=totoc,bibliography=totoc]{report}
\documentclass[headsepline,footsepline,footinclude=false,oneside,fontsize=11pt,paper=a4,listof=totoc,bibliography=totoc,parskip=half]{scrbook}

\usepackage[utf8]{inputenc}
\usepackage[T1]{fontenc}
\usepackage{cascadia-code}
% \usepackage[scaled]{helvet}
\usepackage[sc]{mathpazo}
% \usepackage{newcent}
\usepackage{url}
% \usepackage{cite}
\usepackage[ngerman,american]{babel}
\usepackage[autostyle]{csquotes}
\usepackage[%
  backend=biber,
  url=true,
  style=numeric-comp,
  maxnames=4,
  minnames=3,
  maxbibnames=99,
  % giveninits,
  uniquename=init]{biblatex}
\usepackage{scrhack} % necessary for listings package
\usepackage{listings}
\usepackage{lstautogobble}
\usepackage[pdftex]{graphicx}
% \usepackage[hang,small,bf]{caption}
\usepackage{styles/tum}
\usepackage{tikz}
\usepackage{pgfplots}
\usepackage{pgfplotstable}
\usepackage{amsmath}
\usepackage[USenglish]{datetime2}
\usepackage{setspace}
\usepackage[l3]{csvsimple}
% \usepackage[german,english]{babel}
\usepackage{float}
\usepackage{floatflt}
% \usepackage{fancyhdr}
\usepackage{siunitx}
\usepackage{color}
\usepackage{booktabs}
\usepackage{threeparttable}
\usepackage{subcaption}
\usepackage[final]{microtype}
\usepackage{caption}
\usepackage[pdftex,bookmarks=true,plainpages=false,pdfpagelabels=true,hidelinks]{hyperref}	%TODO make yourself familiar with \label, \ref and \hyperref for referencing figures, tables, chapters, etc.
\usepackage{mdwlist}
\usepackage{enumerate}
\usepackage{array}
\usepackage{longtable}
\usepackage[utf8]{inputenc}
\usepackage[capitalize, noabbrev]{cleveref}
\usepackage{wasysym}
\usepackage{adjustbox}
\usepackage{mdframed}[framemethod=TikZ]
\usepackage[backgroundcolor=TUMLightGray!20,linecolor=TUMAccentOrange,bordercolor=TUMAccentOrange,colorinlistoftodos,tickmarkheight=0.1cm]{todonotes}

% Path for graphics
\graphicspath{{figures/}}

% Include the Thesis metadata like title, author, etc. 
\input{metadata}

\DeclareFieldFormat{urldate}{}

\AtEveryBibitem{
  \clearlist{address}
  \clearfield{date}
  \clearfield{eprint}
  %  \clearfield{doi}
  \clearfield{isbn}
  \clearfield{issn}
  \clearlist{location}
  \clearfield{month}
  \clearfield{labelmonth}
  \clearfield{series}

  \ifentrytype{book}{
    \clearfield{url}
  }
  \ifentrytype{misc}{
  }{
    \clearfield{url}
    \clearlist{publisher}
    \clearname{editor}
  }
}

\bibliography{thesis}

\setkomafont{disposition}{\normalfont\bfseries} % use serif font for headings
\linespread{1.05} % adjust line spread for mathpazo font

\sisetup{round-mode = places, round-precision = 2}
\newcommand{\fixedtwo}{\sisetup{round-mode = places, round-precision = 2}}

% Add table of contents to PDF bookmarks
\BeforeTOCHead[toc]{{\cleardoublepage\pdfbookmark[0]{\contentsname}{toc}}}

% Define TUM corporate design colors
% Taken from http://portal.mytum.de/corporatedesign/index_print/vorlagen/index_farben
\definecolor{TUMBlue}{HTML}{0065BD}
\definecolor{TUMSecondaryBlue}{HTML}{005293}
\definecolor{TUMSecondaryBlue2}{HTML}{003359}
\definecolor{TUMBlack}{HTML}{000000}
\definecolor{TUMWhite}{HTML}{FFFFFF}
\definecolor{TUMDarkGray}{HTML}{333333}
\definecolor{TUMGray}{HTML}{808080}
\definecolor{TUMLightGray}{HTML}{CCCCC6}
\definecolor{TUMAccentGray}{HTML}{DAD7CB}
\definecolor{TUMAccentOrange}{HTML}{E37222}
\definecolor{TUMAccentGreen}{HTML}{A2AD00}
\definecolor{TUMAccentLightBlue}{HTML}{98C6EA}
\definecolor{TUMAccentBlue}{HTML}{64A0C8}

\addto\extrasamerican{
  \def\lstnumberautorefname{Line}
  \def\chapterautorefname{Chapter}
  \def\sectionautorefname{Section}
  \def\subsectionautorefname{Subsection}
  \def\subsubsectionautorefname{Subsubsection}
}

\addto\extrasngerman{
  \def\lstnumberautorefname{Zeile}
}

\mdfsetup{linewidth=1pt, linecolor=TUMDarkGray, nobreak}
\newtheorem{definition}{Definition}
\usetikzlibrary{shapes,fit,positioning,arrows, calc}

% Settings for pgfplots
\pgfplotsset{compat=newest}
\pgfplotsset{
  % For available color names, see http://www.latextemplates.com/svgnames-colors
  cycle list={TUMBlue\\TUMAccentOrange\\TUMAccentGreen\\TUMSecondaryBlue2\\TUMDarkGray\\},
}
\usepgfplotslibrary{statistics}

%%%%%%%%%%%%%%%%%%%%%%%%%%%%%%%%%%%%%%%%%%%%%%%%%%%%%%%%%%%%%%%%%%%%%%%%%%%%%%%%%%%%%%%%%%%%%%%%%
% Custom config for code listings
%%%%%%%%%%%%%%%%%%%%%%%%%%%%%%%%%%%%%%%%%%%%%%%%%%%%%%%%%%%%%%%%%%%%%%%%%%%%%%%%%%%%%%%%%%%%%%%%%
\lstset{%
  numbers=left,
  showstringspaces=false,
  basicstyle=\ttfamily,
  columns=fullflexible,
  autogobble,
  keywordstyle=\bfseries\color{TUMBlue},
  stringstyle=\color{TUMAccentBlue},
  commentstyle=\color{TUMAccentGreen},
  ndkeywordstyle=\color{TUMDarkGray}\bfseries,
  captionpos=b,
  breaklines=true,
  postbreak=\mbox{\textcolor{TUMAccentGray}{$\hookrightarrow$}\space}
}

% !TeX root = ..\thesis.tex
% http://tex.stackexchange.com/questions/152829/how-can-i-highlight-yaml-code-in-a-pretty-way-with-listings
\newcommand\YAMLcolonstyle{\color{TUMDarkGray}\mdseries}
\newcommand\YAMLkeystyle{\color{black}\bfseries}
\newcommand\YAMLvaluestyle{\color{TUMAccentBlue}\mdseries}

\makeatletter

% here is a macro expanding to the name of the language
% (handy if you decide to change it further down the road)
\newcommand\language@yaml{yaml}

\expandafter\expandafter\expandafter\lstdefinelanguage
\expandafter{\language@yaml}
{
  keywords={true,false,null,y,n},
  sensitive=false,
  comment=[l]{\#},
  morecomment=[s]{/*}{*/},
  moredelim=[l][\color{orange}]{\&},
  moredelim=[l][\color{magenta}]{*},
  moredelim=**[il][\YAMLcolonstyle{:}\YAMLvaluestyle]{:},   % switch to value style at :
  morestring=[b]',
  morestring=[b]",
  literate =    {---}{{\ProcessThreeDashes}}3
                {>}{{\textcolor{red}\textgreater}}1     
                % {|}{{\textcolor{red}\textbar}}1 
                {\ -\ }{{\mdseries\ -\ }}3,
}

% switch to key style at EOL
\lst@AddToHook{EveryLine}{\ifx\lst@language\language@yaml\YAMLkeystyle\fi}
\makeatother

\newcommand\ProcessThreeDashes{\llap{\color{cyan}\mdseries-{-}-}}

%%%%%%%%%%%%%%%%%%%%%%%%%%%%%%%%%%%%%%%%%%%%%%%%%%%%%%%%%%%%%%%%%%%%%%%%%%%%%%%%%%%%%%%%%%%%%%%%%
% Custom Commands for this template
%%%%%%%%%%%%%%%%%%%%%%%%%%%%%%%%%%%%%%%%%%%%%%%%%%%%%%%%%%%%%%%%%%%%%%%%%%%%%%%%%%%%%%%%%%%%%%%%%

% Annotate feedback you received 
\newcommand{\feedback}[1]{\todo[inline,bordercolor=TUMAccentGreen]{#1}}
\newcommand{\comment}[1]{\todo[bordercolor=TUMAccentLightBlue,linecolor=TUMAccentLightBlue]{#1}}

% State what is missing in this spot
\newcommand{\missing}[1]{\todo[inline,color=yellow,bordercolor=TUMGray,linecolor=TUMGray]{#1}}

% Inline to do note: 
\newcommand{\TODO}[1]{\todo[inline]{#1}}

\newcommand{\addref}{\todo[bordercolor=red,linecolor=red]{Add reference.}}
\newcommand{\rewrite}[1]{\todo[bordercolor=TUMAccentBlue,linecolor=TUMAccentBlue]{#1}}


%%%%%%%%%%%%%%%%%%%%%%%%%%%%%%%%%%%%%%%%%%%%%%%%%%%%%%%%%%%
% Theses specific packages go here
%%%%%%%%%%%%%%%%%%%%%%%%%%%%%%%%%%%%%%%%%%%%%%%%%%%%%%%%%%%
\usepackage[printonlyused]{acronym}


%%%%%%%%%%%%%%%%%%%%%%%%%%%%%%%%%%%%%%%%%%%%%%%%%%%%%%%%%%%
% Begin of document
%%%%%%%%%%%%%%%%%%%%%%%%%%%%%%%%%%%%%%%%%%%%%%%%%%%%%%%%%%%

\begin{document}
% \setlength{\evensidemargin}{22pt}
% \setlength{\oddsidemargin}{22pt}


\hypersetup{pdfborder={0 0 0}, pdfauthor={\author}, pdftitle={\title}}

\lstset{showspaces=false, numbers=left, frame=single, basicstyle=\small}

\pagenumbering{alph}

%------- Cover and Title setup -------
\begin{titlepage}
    % HACK for two-sided documents: ignore binding correction for cover page.
    % Adapted from Markus Kohm's KOMA-Script titlepage=firstiscover handling.
    % See http://mirrors.ctan.org/macros/latex/contrib/koma-script/scrkernel-title.dtx,
    % \maketitle macro.
    \oddsidemargin=\evensidemargin\relax
    \textwidth=\dimexpr\paperwidth-2\evensidemargin-2in\relax
    \hsize=\textwidth\relax

    \centering
    \oTUM{4cm}

    \vspace{5mm}
    {\huge\MakeUppercase{School of Computation, Information and Technology \\ -- Informatics --} \par }

    \vspace{5mm}
    {\large\MakeUppercase{Technical University of Munich}\par}

    \vspace{15mm}
    {\Large {\degree}'s Thesis in \program \par}

    \vspace{10mm}
    {\huge\bfseries \title \par}

    \vspace{10mm}
    {\LARGE \author}


\end{titlepage}

\frontmatter{}
\begin{titlepage}
        \centering

        \oTUM{4cm}

        \vspace{5mm}
        {\huge\MakeUppercase{School of Computation, Information and Technology \\ -- Informatics --} \par }

        \vspace{5mm}
        {\large\MakeUppercase{Technical University of Munich}}

        \vspace{10mm}
        {\Large {\degree}'s Thesis in \program }

        \vspace{10mm}
        {\LARGE\bfseries \title \par}

        \vspace{5mm}
        {\LARGE\bfseries \foreignlanguage{ngerman}{\titleGer} \par}

        \vspace{10mm}
        \begin{tabular}{l l}
                Author:          & \author         \\
                Supervisor:      & \supervisor     \\
                Advisor:         & \advisor        \\
                Advisor:         & \advisorTwo        \\
                Submission Date: & \date \\
        \end{tabular}
\end{titlepage}


%------- Disclaimer -------
\clearpage
\thispagestyle{empty}
\vspace*{0.8\textheight}
\noindent
I confirm that this \MakeLowercase{\degree}'s thesis is my own work and I have documented all sources and material used.

\vspace{15mm}
\noindent
Munich, \date \hspace{\stretch{1}} \author
\newpage



%------- Acknowledgements -------
% \newpage
% \thispagestyle{empty}
% \mbox{}
% \chapter*{Acknowledgements}




% \pagenumbering{roman}

%------- Abstracts -------
% \selectlanguage{english}
\addchap{Abstract}

Software testing plays a critical role in ensuring system quality, reliability, and performance. 
However, the effectiveness of testing is limited by the presence of flaky tests, which are tests that can pass or fail randomly with the same version of the \acf*{sut} and configuration.
\acf*{e2e} tests in particular can be susceptible to flaky behavior, posing significant challenges to the software development process. 
While existing research has primarily focused on the detection of flaky unit tests, this thesis addresses the detection of flaky failures in \acs*{e2e} tests. 
We adapt a technique from \citeauthor*{bell_deflaker_2018} for detecting flaky Java unit test failures and apply it to \acs*{e2e} tests, which involve multiple programming languages and a different context for test execution. Our approach requires the instrumentation of code to collect coverage data during test execution. 
To evaluate our methodology, we selected two open source projects, \textsc{ArTEMiS} and \textsc{n8n}, which use the Cypress framework for \acs*{e2e} testing.

Our contributions include a novel methodology for detecting flaky failures in \acs*{e2e} testing, an evaluation of our approach on two open source projects, and guidelines for practitioners. 
Our evaluation shows that instrumentation can have a significant impact on test execution, and that our approach correctly identified flaky failures 42\% of the time. 
73\% of the failures identified by our approach were indeed flaky, and 27\% were false positives.
The results indicate that while the adapted approach has potential, it is not directly applicable to \acs*{e2e} tests, and further research is needed to improve its performance. 
This is due to the high implementation complexity of the approach, the impact of instrumentation on test execution, and the low recall of the approach.
Future work should also explore the applicability of our approach to different technology stacks, extend the coverage collection tool to support other test frameworks, and investigate the impact of instrumentation on test execution.
% \clearpage
% \selectlanguage{german}
\chapter{Abstract}

Software-Tests sind ein entscheidender Aspekt der Softwareentwicklung, um die Qualität und Zuverlässigkeit von Systemen sicherzustellen. Obwohl Testfälle eigentlich deterministisch sein sollten, können sogenannte "flaky" Tests nicht-deterministisches Verhalten aufweisen, was zu intermittierenden Fehlern führt, ohne dass Änderungen am zu testenden System (SUT) oder am Test selbst vorgenommen wurden. Diese Arbeit zielt darauf ab, flaky Fehler in End-to-End (E2E)-Tests zu erkennen, indem die von Bell et al. in "DeFlaker" vorgeschlagene Methode zum Erkennen von flaky Fehlern in Unit-Tests angewendet wird. Durch die Konzentration auf flaky Fehler anstelle von flaky Tests soll unser Ansatz während der Softwareentwicklung im Continuous Integration (CI)-Prozess einen höheren Mehrwert bieten.

Wir schlagen eine Methodik zur Erkennung von flaky Fehlern in E2E-Tests vor und bewerten deren Leistung anhand von zwei Open-Source-Projekten als Fallstudien: Artemis, eine webbasierte Lernplattform für Programmierübungen, und n8n, ein Workflow-Automatisierungstool. Beide Projekte verwenden Cypress für E2E-Tests. Unsere Bewertung beurteilt die Auswirkungen unserer Methodik auf die Testausführung und ihre Fähigkeit, flaky Fehler korrekt zu identifizieren. Unser Ansatz funktioniert gut und identifiziert flaky Fehler in 90\% der Fälle korrekt, wobei es in drei Fällen falsche Positive gab, bei denen unser Ansatz einen Fehler als flaky einstufte, der es jedoch nicht war.

Diese Arbeit bietet einen Überblick über E2E-Tests, flaky Tests, Erfassung von Codeabdeckung und alternative Ansätze zur Erkennung von flaky Tests. Sie beschreibt auch die Methodik, Fallstudien, Ergebnisse der Bewertung, Einschränkungen und zukünftige Arbeiten zur Erkennung von flaky Fehlern in E2E-Tests.

% \clearpage
% \selectlanguage{english}

%------- Table of contents -------
\microtypesetup{protrusion=false}
\tableofcontents{}
\microtypesetup{protrusion=true}

%------- List of todos -------
\todototoc
\listoftodos

\mainmatter{}
% \pagenumbering{arabic}

% \fancyhead{}
% \pagestyle{fancy}
% \fancyhead[LE]{\slshape \leftmark}
% \fancyhead[RO]{\slshape \rightmark}
% \headheight=15pt


\chapter*{Not yet assigned ideas}
\TODO{Talk about the limited amount of actual failures in both projects for code that is run by the CI}
\TODO{Overview of the artemis historic data and insight, that all tested fails did eventually pass}
\TODO{Mention the significant effect cypress rerunning had on artemis}
\TODO{Elaborate how this approach is useful in everyday development contrary to flaky test detection}
\TODO{Talk about the side-effect that failures can be located more easily}
\TODO{Limitation: does not work if serivice does not start}
\TODO{Problems: include bigger issues during development}

%------- chapter 1 -------

\chapter{Introduction}\label[chapter]{introduction}
Software testing is an integral part of the software development process that ensures a system's quality and dependability.
With the execution of test cases, the goal of software testing is to detect software bugs.
It is assumed that test cases are be deterministic, meaning that given the same input, they would always yield the same output.
In practice, however, test cases can be non-deterministic or \emph{flaky} \autocite{luo_empirical_2014} meaning they can fail intermittently without any modification to the \ac{sut} or the test itself.
\Ac{ui} tests, which frequently involve asynchronous activities and limited computer resources, might be especially susceptible to flaky behaviour \autocite{romano_empirical_2021}.

Flaky tests can waste resources, increase development time and erode confidence in the testing procedure.
Google reports that about 16\% of thier tests exhibit some level of flakiness \autocite{micco_state_2017}.
Even given the importance of testing, the detection of flaky tests is still a challenge. 
The most common approach to identify flaky tests is rerunning \autocite{lam_idflakies_2019}.
However, rerunning has several downsides, such as consuming significant computational resources, generating false positives, and potentially missing certain types of flaky tests \autocite{bell_deflaker_2018, luo_empirical_2014}.
\feedback{The paragraph does not really describe the point of the thesis, which is to detect flaky failures. Maybe move from flaky tests to flaky failures?}

An important distinction that has to be made is the difference between flaky tests and flaky test failures.
While most research has focused on detecting flaky tests, it is important to distinguish them from flaky failures.
\Citeauthor*{haben_importance_2023} find that \enquote{approximately 76.2\% of all regression faults} \autocite{haben_importance_2023} are missed when discarding faults of flaky tests.
Instead we want to focus on detecting flaky failures in this thesis to provide more value to the \ac{ci} during software development.
\todo{Extend this paragraph?}

To achieve this for \ac{e2e} tests, we focus on identifying flaky failures by applying a technique for detecting flaky unit test failures proposed by \citeauthor*{bell_deflaker_2018} in \citetitle{bell_deflaker_2018} \autocite{bell_deflaker_2018} to \ac{e2e} tests.
\Citeauthor*{bell_deflaker_2018} that that it is possible to identify if a failure is flaky by comparing the code covered by the test with the changes since the last passing run. To evaluate our approach we selected two open source projects as case studies.
\textsc{Artemis} \autocite{krusche_artemis_2018} is a web-based learning platform for programming exercises.
It uses Java for server code and JavaScript for the client-side code. 
\textsc{n8n} \autocite{noauthor_n8n_2023} is a workflow automation tool written in TypeScript for both server and client.
Both projects use Cypress \autocite{noauthor_cypress-iocypress_2023} for \ac{e2e} testing.

The contributions of this thesis are:
\begin{itemize}
	\item \textbf{Methodology.} We propose a methodology for detecting flaky failures in \ac{e2e} tests based on previous work in detecting flaky unit test failures \autocite{bell_deflaker_2018}.
	\item \textbf{Case Studies.} We evaluate our approach on two open source projects by analyzing the impact on execution our method has and evaluating the performance of our approach.
	      During the evaluation we saw that the instrumentation can have significant impact on test execution.
	      Our approach performed well by correctly ident flaky failures in 90\% \todo{Correct number} of the cases.
	      During the evaluation we saw three \todo{Correct number} false positives where our approach identified a failure as flaky, but it was not.
	      \todo{Add more info on study and results}
	\item \textbf{Guidelines.} \todo{Add guidelines}
\end{itemize}

The remainder of this thesis is organized as follows:
In \cref{background} we will give some background information on the topic of \ac{e2e} testing, flaky tests, and coverage collection.
We will review current alternative approaches in \cref{related_work}.
In \cref{methodology} we will describe the methodology used in the project, including the instrumentation, data collection and environment setup.
In \cref{evaluation} we will present the results of the project and answer our research questions.
In \cref{threats} we will discuss the limitations of the project and in \cref{conclusion} we will summarize results of the evaluation and discuss future work.



%------- chapter 2 -------

\chapter{Background}\label[chapter]{background}

This chapter provides essential background information for understanding the context of this thesis.

\section{E2E Testing}

\TODO{General rouch introduction to software testing before moving to e2e and ui only}
\TODO{Explanin what E2E testing is and why it is important}
\TODO{Give an overview of testing frameworks available (also not for browsers?)}
\TODO{Explain what Cypress is}

\section{Flaky Tests}

\TODO{Mention the root causes for flaky tests and failures}
\TODO{Press the importance and usefulness of identiffying flaky failures instead of just flaky tests}
\TODO{Explain why this approach with flaky failures is superior to identifying flaky tests (more common)}
\TODO{Explain what flaky failures are and why they are a problem}

\section{Coverage Collection}

\TODO{Explain what coverage collection is}
\feedback{Coverage collection is expensive?}
\TODO{Mention some usecases for coverage data}

%------- chapter 3 -------

\chapter{Related Work}\label[chapter]{related_work}

\TODO{Give an overview of related work}

\section{Flaky Failure Detection}

\TODO{Look at some papers that detect flaky tests}
\TODO{Explain why flaky test detection is not sufficient but we need flaky failure detection}
\feedback{Fabian Leinen: Den eigenen Gedanken in dem Kapitel möglichst meiden. Z.B. in "The Importance of Discerning Flaky from Fault-triggering Test Failures: A Case Study on the Chromium CI" ist das ja wunderschön dargelegt.
}

\subsection{Expaning on \emph{DeFlaker}}

\TODO{Explain how we can use the same approach for E2E tests}

\section{Usage of coverage data in testing}

\TODO{Look at some papers that use coverage data}
\TODO{Look at test case selection based on coverage data}

%------- chapter 4 -------

\chapter{Methodology}\label[chapter]{methodology}

\TODO{Overview for the Chapter and introduction to the different parts of the project.}
\TODO{Explain the approach in DeFlaker and how this project has different requirements?}

\section{Instrumentation}
\TODO{Info about issues with default instrumentation which is not designed for remote collection for specific test cases.}
\TODO{Explain why the entire coverage is collected and not just change coverage}
\TODO{Explain how the decision was made on which level to collect coverage}
\TODO{Explain how the coverage is matched to test cases: propagation vs timing}
\TODO{Mention the possible impact of insturmentation on the behaviour of the application and the tests.}

\section{Change Collection}


\section{Implementation}
\TODO{Explain the implementation of the coverage collection plugins}
\TODO{Explain the implementation of the analysis program}
\begin{figure}[h]
	\centering
	\begin{tikzpicture}[node distance=2cm]
		% Nodes
		\node (cypress) [rectangle, draw] {Cypress};
		\node (tests) [rectangle, draw, right of=cypress, xshift=2cm] {Specs};
		\node (plugin) [rectangle, draw, below of=cypress] {Coverage Collection Plugin};
		\node (client) [rectangle, draw, below left of=plugin, xshift=-2cm] {Client};
		\node (server) [rectangle, draw, below right of=plugin, xshift=2cm] {Server};
		\node (coverage) [rectangle, draw, below of=plugin] {Full Coverage};
		\node (analysis) [rectangle, draw, below of=coverage] {Analysis Program};

		% Edges
		\draw [<-] (tests) --  node[pos=0.5, below] {runs} (cypress);
		\draw [-] (cypress) -- (plugin);
		\draw [<->] (plugin) -- (client);
		\draw [<->] (plugin) -- (server);
		% \draw [->] (client) |- (coverage);
		% \draw [->] (server) |- (coverage);
		\draw [->] (plugin) -- node[pos=0.5, below] {on fail} (coverage);
		\draw [->] (coverage) -- (analysis);
	\end{tikzpicture}
	\caption[Full stack coverage collection process in e2e testing]{Diagram showing the full stack coverage collection process in \ac{e2e} testing}
	\label[figure]{full_stack_coverage_collection}
\end{figure}

\subsection{Coverage Collection Plugins}
\subsubsection{Java server}
\TODO{Explain the use of jacoco and it's setup in server mode}
\TODO{Transform coverage to XML and to JSON}
\subsubsection{Node.js server}
\TODO{Explain NYC cannot be used as dynamically and would need a reset mechanism}
\TODO{Explain the use of the node debugging protocol to use V8 coverage}
\subsubsection{Client}
\TODO{Elaboration of instrumemtation ideas that did not work out. Preprocession with istanbul as suggested by cypress and teamscale.}
\TODO{Elaboration of chrome native based coverage}
\TODO{Small tangent on electron insturmentation which is special as cypress already runs in an electron process}

\subsection{Analysis Program}
\TODO{Explain how to get file based changes from the git log}
\TODO{Explain how to get line based changes from the git diff and the issues with buffer size}

%------- chapter 5 -------

\chapter{Evaluation}\label[chapter]{evaluation}

\section{Research Questions}
\begin{enumerate}
	\item[\textbf{RQ1:}] How effective is the adapted Deflaker approach in identifying flaky failures in UI end-to-end tests?
	\item[\textbf{RQ2:}] To what extent does instrumentation impact the temporal behavior of the application during end-to-end testing?
	\item[\textbf{RQ3:}] How does the instrumentation of coverage collection affect the failure rate of end-to-end tests in the studied projects?
\end{enumerate}

\section{Case Studies}
\TODO{Bigger introduction to artemis and n8n}
\TODO{Artemis was chosen because it is a large project with a lot of tests and a CI pipeline that experienced issues with flaky tests at the university}
\TODO{Galdys was a contender but tests did not show any flakiness and there were also no indication in the commit messages}
\TODO{Additional project, ToolJet turned out to have a too complicated setup for their E2E tests}
\TODO{n8n was chosen from the list of projects with cypress and JavaScript and had issues with flaky tests before}

\section{Evaluation Setup}
\TODO{Setup as part of the existing CI pipeline}
\TODO{Talk about issues with running artemis tests locally, too much setup needed and too much environment}

\subsection{Artemis (Bamboo)}
\TODO{Explain the setup of the Artemis CI pipeline}
\TODO{Live and historic evaluation}
\TODO{Elaborate issues with the historic evaluation due to CI and code changes that had to be backported}
\TODO{Explain setup of live evaluation with duplicated plan}
\TODO{Specialty of the Artemis CI pipeline, regular e2e tests as base}

\subsection{n8n (GitHub)}
\TODO{Explain the setup of the n8n CI pipeline}
\TODO{Fork for historic evaluation}
\TODO{Issues with GithHub Runners due to memory limitations}
\TODO{Issues with tests after switching to chrome, exclusion of tests that did not pass in three runs}
\TODO{Setup of historic evaluation with baseline and comparisons}
\TODO{Explain the timing baseline collected on hosted runners}
\feedback{Fabian Leinen: Zusammenfassend gibt es ja drei Subjects zum Evaluieren: Artemis Live, Artemis Historic, n8n historic. Da musst du glaube ich drauf achten, dass es hier so klar wird und in der Results Section muss klar sein, was du womit beantwortest.}

\section{Evaluation Results}

\subsection{RQ1: Does instrumentation change the temporal behavior of the application?}
\TODO{Review the gathered data for both case studies}
You can review the full table of testcases that failed during our evaluation in \cref{behaviour_change_results}.

\subsection{RQ2: Does instrumentation change the failure rate of tests?}
\begin{table}[h]
	\centering
	\csvautobooktabular{data/runResults.csv}
	\caption{Statistics of the evaluation runs}
	\label[table]{runResults}
\end{table}

\begin{table}[h]
	\centering
	\csvautobooktabular{data/testcaseResults.csv}
	\caption{Statistics of testcase results during evaluation}
	\label[table]{testResults}
\end{table}
\TODO{Review the gathered data for both case studies}

\subsection{RQ3: What are the precision and recall of the failure categorizer?}
\TODO{Review the gathered data for both case studies}

\subsection{Additional Observations}
\TODO{Issues with memory}
\feedback{Fabian Leinen: Was kommt alles in diese Subsection? Mein Gefühl ist, dass es nicht viel ist und das auch ganz gut bei RQ1 oder Discussion rein passt.}

\section{Discussion}
\TODO{Give an overview of the results}
\feedback{Dies Discussion kann wie hier in einer eigenen Section sein, Dann achte bloß bei den Results etwas drauf nicht zu sehr Interpretationen und so weiter einzubauen sondern möglichst trocken die (quantitativen) Ergebnisse zu präsentieren.}

\subsection{Quality of Failure Categorization}
\TODO{Interpret the results of RQ1}

\subsection{Change of Behaviour caused by Instrumentation}
\TODO{Interpret the results of RQ2}
\TODO{Show the inpact on RQ3}
\TODO{Interprent the results of RQ2 for both case studies}

%------- chapter 6 -------

\chapter{Threats to Validity}\label[chapter]{threats}
\TODO{Write chapter introduction}

\section{Threats to Internal Validity}
\TODO{Dependence on the CI pipeline}
\TODO{Missing confirmed failures}
\TODO{Limited insights to non code changes and testcase changes}

\section{Threats to External Validity}
\TODO{Only two case studies}
\TODO{Limited availablity of run labels}


%------- chapter 7 -------

\chapter{Conclusion}\label[chapter]{conclusion}

\TODO{Summary of the results}
\TODO{Summary of the implementation and evaluation}


\section{Future Work}

\TODO{Replicate tests with other test frameworks such as Playwright}
\TODO{Extend analysis to other languages}
\TODO{Investigate ways to mitigate the impact of instrumentation on test execution}
\TODO{Make the analysis more robust for non code changes and test case changes}

\appendix

\chapter[Behaviour change results]{Extensive results for the behaviour change evaluation}\label[appendix]{behaviour_change_results}

\csvautobooktabular{data/runResults.csv}

\begin{adjustbox}{max width=\textwidth, caption={Statistics of testcase results during evaluation}, float=table}
	\begin{tabular}{@{}lrrrrr@{}} \toprule
		\multicolumn{2}{c}{} & \multicolumn{2}{c}{Uninstrumented} & \multicolumn{2}{c}{Instrumented}                                                       \\ \cmidrule{3-4} \cmidrule{5-6}
		Test name            & Duration increase \%               & passed                           & failed  & passed              & failed              \\ \midrule
		\csvreader[
			head to column names
		]{data/testcaseStats.csv}{}{%
		\name                & \durationIncrease                  & \passed                          & \failed & \passedInstrumented & \failedInstrumented \\
		}
		\\ \bottomrule
	\end{tabular}
\end{adjustbox}

\microtypesetup{protrusion=false}
% !TeX root = ..\thesis.tex

\addchap{Acronyms}
\begin{acronym}
    \itemsep-.25\baselineskip
    \acro{api}[API]{Application Programming Interface}
    \acro{ci}[CI]{Continuous Integration}
    \acro{cli}[CLI]{Command Line Interface}
    \acro{dom}[DOM]{Document Object Model}
    \acro{e2e}[E2E]{End-to-End}
    \acro{fn}[FN]{False Negative}
    \acro{fp}[FP]{False Positive}
    \acro{gui}[GUI]{Graphical User Interface}
    \acro{mcc}[MCC]{Matthews Correlation Coefficient}
    \acro{pr}[PR]{Pull Request}
    \acro{rq}[RQ]{Research Question}
    \acro{sut}[SUT]{System Under Test}
    \acro{tn}[TN]{True Negative}
    \acro{tp}[TP]{True Positive}
    \acro{tum}[TUM]{Technical University of Munich}
    \acro{ui}[UI]{User Interface}
    \acro{od}[OD]{order-dependent}
    \acro{nod}[NOD]{non-order-dependent}
\end{acronym}

\listoffigures{}
\listoftables{}
\microtypesetup{protrusion=true}
\printbibliography{}

\end{document}
