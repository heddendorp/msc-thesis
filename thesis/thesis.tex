% Add common preamble to the document
% !TeX root = ..\thesis.tex
% \documentclass[a4paper,12pt,twoside,listof=totoc,bibliography=totoc]{report}
\documentclass[headsepline,footsepline,footinclude=false,oneside,fontsize=11pt,paper=a4,listof=totoc,bibliography=totoc]{scrbook}

\usepackage[utf8]{inputenc}
\usepackage[T1]{fontenc}
\usepackage{cascadia-code}
% \usepackage[scaled]{helvet}
\usepackage[sc]{mathpazo}
% \usepackage{newcent}
\usepackage{url}
% \usepackage{cite}
\usepackage[ngerman,american]{babel}
\usepackage[autostyle]{csquotes}
\usepackage[%
  backend=biber,
  url=true,
  style=numeric,
  maxnames=4,
  minnames=3,
  maxbibnames=99,
  giveninits,
  uniquename=init]{biblatex}
\usepackage{scrhack} % necessary for listings package
\usepackage{listings}
\usepackage{lstautogobble}
\usepackage[pdftex]{graphicx}
% \usepackage[hang,small,bf]{caption}
\usepackage{styles/tum}
\usepackage{tikz}
\usepackage{setspace}
\usepackage[l3]{csvsimple}
% \usepackage[german,english]{babel}
\usepackage{float}
\usepackage{floatflt}
% \usepackage{fancyhdr}
\usepackage{color}
\usepackage{booktabs}
\usepackage[final]{microtype}
\usepackage{caption}
\usepackage[pdftex,bookmarks=true,plainpages=false,pdfpagelabels=true,hidelinks]{hyperref}	%TODO make yourself familiar with \label, \ref and \hyperref for referencing figures, tables, chapters, etc.
\usepackage{mdwlist}
\usepackage{enumerate}
\usepackage{array}
\usepackage{longtable}
\usepackage[utf8]{inputenc}
\usepackage[capitalize, noabbrev]{cleveref}
\usepackage{wasysym}
\usepackage{adjustbox}
\usepackage[backgroundcolor=TUMLightGray!20,linecolor=TUMAccentOrange,bordercolor=TUMAccentOrange,colorinlistoftodos]{todonotes}

% Path for graphics
\graphicspath{{figures/}}

% Include the Thesis metadata like title, author, etc. 
\input{metadata}

\bibliography{thesis}

\setkomafont{disposition}{\normalfont\bfseries} % use serif font for headings
\linespread{1.05} % adjust line spread for mathpazo font


% Add table of contents to PDF bookmarks
\BeforeTOCHead[toc]{{\cleardoublepage\pdfbookmark[0]{\contentsname}{toc}}}

% Define TUM corporate design colors
% Taken from http://portal.mytum.de/corporatedesign/index_print/vorlagen/index_farben
\definecolor{TUMBlue}{HTML}{0065BD}
\definecolor{TUMSecondaryBlue}{HTML}{005293}
\definecolor{TUMSecondaryBlue2}{HTML}{003359}
\definecolor{TUMBlack}{HTML}{000000}
\definecolor{TUMWhite}{HTML}{FFFFFF}
\definecolor{TUMDarkGray}{HTML}{333333}
\definecolor{TUMGray}{HTML}{808080}
\definecolor{TUMLightGray}{HTML}{CCCCC6}
\definecolor{TUMAccentGray}{HTML}{DAD7CB}
\definecolor{TUMAccentOrange}{HTML}{E37222}
\definecolor{TUMAccentGreen}{HTML}{A2AD00}
\definecolor{TUMAccentLightBlue}{HTML}{98C6EA}
\definecolor{TUMAccentBlue}{HTML}{64A0C8}

\addto\extrasamerican{
	\def\lstnumberautorefname{Line}
	\def\chapterautorefname{Chapter}
	\def\sectionautorefname{Section}
	\def\subsectionautorefname{Subsection}
	\def\subsubsectionautorefname{Subsubsection}
}

\addto\extrasngerman{
	\def\lstnumberautorefname{Zeile}
}

%%%%%%%%%%%%%%%%%%%%%%%%%%%%%%%%%%%%%%%%%%%%%%%%%%%%%%%%%%%%%%%%%%%%%%%%%%%%%%%%%%%%%%%%%%%%%%%%%
% Custom config for code listings
%%%%%%%%%%%%%%%%%%%%%%%%%%%%%%%%%%%%%%%%%%%%%%%%%%%%%%%%%%%%%%%%%%%%%%%%%%%%%%%%%%%%%%%%%%%%%%%%%
\lstset{%
  numbers=left,
  showstringspaces=false,
  basicstyle=\ttfamily,
  columns=fullflexible,
  autogobble,
  keywordstyle=\bfseries\color{TUMBlue},
  stringstyle=\color{TUMAccentBlue},
  commentstyle=\color{TUMAccentGreen},
  ndkeywordstyle=\color{TUMDarkGray}\bfseries,
  captionpos=b,
  breaklines=true,
  postbreak=\mbox{\textcolor{TUMAccentGray}{$\hookrightarrow$}\space}
}

\input{styles/yaml_listing.tex}

%%%%%%%%%%%%%%%%%%%%%%%%%%%%%%%%%%%%%%%%%%%%%%%%%%%%%%%%%%%%%%%%%%%%%%%%%%%%%%%%%%%%%%%%%%%%%%%%%
% Custom Commands for this template
%%%%%%%%%%%%%%%%%%%%%%%%%%%%%%%%%%%%%%%%%%%%%%%%%%%%%%%%%%%%%%%%%%%%%%%%%%%%%%%%%%%%%%%%%%%%%%%%%

% Annotate feedback you received 
\newcommand{\feedback}[1]{\todo[inline,bordercolor=TUMAccentGreen]{#1}}
\newcommand{\comment}[1]{\todo[bordercolor=TUMAccentLightBlue,linecolor=TUMAccentLightBlue]{#1}}

% State what is missing in this spot
\newcommand{\missing}[1]{\todo[inline,color=yellow,bordercolor=TUMGray,linecolor=TUMGray]{#1}}

% Inline to do note: 
\newcommand{\TODO}[1]{\todo[inline]{#1}}

\newcommand{\addref}{\todo[bordercolor=red,linecolor=red,tickmarkheight=0.1cm]{Add reference.}}
\newcommand{\rewrite}[1]{\todo[bordercolor=TUMAccentBlue,linecolor=TUMAccentBlue]{#1}}


%%%%%%%%%%%%%%%%%%%%%%%%%%%%%%%%%%%%%%%%%%%%%%%%%%%%%%%%%%%
% Theses specific packages go here
%%%%%%%%%%%%%%%%%%%%%%%%%%%%%%%%%%%%%%%%%%%%%%%%%%%%%%%%%%%
\usepackage[printonlyused]{acronym}


%%%%%%%%%%%%%%%%%%%%%%%%%%%%%%%%%%%%%%%%%%%%%%%%%%%%%%%%%%%
% Begin of document
%%%%%%%%%%%%%%%%%%%%%%%%%%%%%%%%%%%%%%%%%%%%%%%%%%%%%%%%%%%

\begin{document}
\setlength{\evensidemargin}{22pt}
\setlength{\oddsidemargin}{22pt}


\hypersetup{pdfborder={0 0 0}, pdfauthor={\author}, pdftitle={\title}}

\lstset{showspaces=false, numbers=left, frame=single, basicstyle=\small}

\pagenumbering{alph}

%------- Cover and Title setup -------
\begin{titlepage}
    % HACK for two-sided documents: ignore binding correction for cover page.
    % Adapted from Markus Kohm's KOMA-Script titlepage=firstiscover handling.
    % See http://mirrors.ctan.org/macros/latex/contrib/koma-script/scrkernel-title.dtx,
    % \maketitle macro.
    \oddsidemargin=\evensidemargin\relax
    \textwidth=\dimexpr\paperwidth-2\evensidemargin-2in\relax
    \hsize=\textwidth\relax

    \centering
    \oTUM{4cm}

    \vspace{5mm}
    {\huge\MakeUppercase{School of Computation, Information and Technology \\ -- Informatics --} \par }

    \vspace{5mm}
    {\large\MakeUppercase{Technical University of Munich}\par}

    \vspace{15mm}
    {\Large {\degree}'s Thesis in \program \par}

    \vspace{10mm}
    {\huge\bfseries \title \par}

    \vspace{10mm}
    {\LARGE \author}


\end{titlepage}

\begin{titlepage}
        \centering

        \oTUM{4cm}

        \vspace{5mm}
        {\huge\MakeUppercase{School of Computation, Information and Technology \\ -- Informatics --} \par }

        \vspace{5mm}
        {\large\MakeUppercase{Technical University of Munich}}

        \vspace{10mm}
        {\Large {\degree}'s Thesis in \program }

        \vspace{10mm}
        {\LARGE\bfseries \title \par}

        \vspace{5mm}
        {\LARGE\bfseries \foreignlanguage{ngerman}{\titleGer} \par}

        \vspace{10mm}
        \begin{tabular}{l l}
                Author:          & \author               \\
                Supervisor:      & \supervisor           \\
                Advisors:        & \advisor, \advisorTwo \\
                Submission Date: & \date                 \\
        \end{tabular}
\end{titlepage}


%------- Disclaimer -------
\newpage
\thispagestyle{empty}
\mbox{}
\include{thesis_tex/disclaimer}

%------- Acknowledgements -------
% \newpage
% \thispagestyle{empty}
% \mbox{}
% \include{thesis_tex/acknowledgement}

\pagenumbering{roman}

%------- Abstracts -------
\selectlanguage{english}
\chapter{Abstract}

Software testing plays a vital role in ensuring system quality, reliability, and performance. 
\acf*{e2e} tests, in particular, can be susceptible to flaky behavior, causing significant challenges to the software development process. 
While existing research has primarily focused on detecting flaky tests, this thesis addresses the detection of flaky failures in \ac*{e2e} tests. 
We adapt a technique from \citeauthor*{bell_deflaker_2018} for detecting flaky Java unit test failures and apply it to \ac*{e2e} tests, which involve multiple languages and a different context for test execution. Our approach requires the instrumentation of code to collect coverage data during test execution. 
To evaluate our methodology, we selected two open-source projects, \textsc{ArTEMiS} and \textsc{n8n}, which use the Cypress framework for \ac*{e2e} testing.

Our contributions include a novel methodology for detecting flaky failures in \ac*{e2e} testing, an evaluation of our approach on two open-source projects, and guidelines for practitioners. 
Our evaluation shows that instrumentation can significantly impact test execution and that our approach correctly identified flaky failures 42\% of the time \todo{Update number after eval finishes}. 
We also observed three false positives in the evaluation. 
The results indicate that while the adapted approach has potential, it is not directly applicable to \ac*{e2e} tests, and further research is needed to improve its performance. 
Future work should explore the applicability of our approach to different tech stacks, extend the coverage collection tool to support other testing frameworks, and investigate the impact of instrumentation on test execution.
\clearpage
% \selectlanguage{german}
% \chapter{Abstract}

Software-Tests sind ein entscheidender Aspekt der Softwareentwicklung, um die Qualität und Zuverlässigkeit von Systemen sicherzustellen. Obwohl Testfälle eigentlich deterministisch sein sollten, können sogenannte "flaky" Tests nicht-deterministisches Verhalten aufweisen, was zu intermittierenden Fehlern führt, ohne dass Änderungen am zu testenden System (SUT) oder am Test selbst vorgenommen wurden. Diese Arbeit zielt darauf ab, flaky Fehler in End-to-End (E2E)-Tests zu erkennen, indem die von Bell et al. in "DeFlaker" vorgeschlagene Methode zum Erkennen von flaky Fehlern in Unit-Tests angewendet wird. Durch die Konzentration auf flaky Fehler anstelle von flaky Tests soll unser Ansatz während der Softwareentwicklung im Continuous Integration (CI)-Prozess einen höheren Mehrwert bieten.

Wir schlagen eine Methodik zur Erkennung von flaky Fehlern in E2E-Tests vor und bewerten deren Leistung anhand von zwei Open-Source-Projekten als Fallstudien: Artemis, eine webbasierte Lernplattform für Programmierübungen, und n8n, ein Workflow-Automatisierungstool. Beide Projekte verwenden Cypress für E2E-Tests. Unsere Bewertung beurteilt die Auswirkungen unserer Methodik auf die Testausführung und ihre Fähigkeit, flaky Fehler korrekt zu identifizieren. Unser Ansatz funktioniert gut und identifiziert flaky Fehler in 90\% der Fälle korrekt, wobei es in drei Fällen falsche Positive gab, bei denen unser Ansatz einen Fehler als flaky einstufte, der es jedoch nicht war.

Diese Arbeit bietet einen Überblick über E2E-Tests, flaky Tests, Erfassung von Codeabdeckung und alternative Ansätze zur Erkennung von flaky Tests. Sie beschreibt auch die Methodik, Fallstudien, Ergebnisse der Bewertung, Einschränkungen und zukünftige Arbeiten zur Erkennung von flaky Fehlern in E2E-Tests.

% \clearpage
% \selectlanguage{english}

%------- Table of contents -------
\tableofcontents
\clearpage

\clearpage

%------- Common Acronyms -------
% !TeX root = ..\thesis.tex

\addchap{Abbreviations}
\begin{acronym}
    \itemsep-.25\baselineskip
    \acro{gui}[GUI]{Graphical User Interface}
    \acro{ui}[UI]{User Interface}
    \acro{e2e}[E2E]{End-to-End}
    \acro{ci}[CI]{Continuous Integration}
    \acro{sut}[SUT]{System Under Test}
\end{acronym}

\pagenumbering{arabic}

\fancyhead{}
\pagestyle{fancy}
\fancyhead[LE]{\slshape \leftmark}
\fancyhead[RO]{\slshape \rightmark}
\headheight=15pt




%------- chapter 1 -------

\chapter{Introduction}\label[chapter]{introduction}
Software testing is an integral part of the software development process that ensures a system's quality and dependability. With the execution of test cases, one of the primary goals of software testing is to detect software bugs or errors. In a perfect world, test cases would be deterministic, meaning that given the same input, they would always yield the same output. In practice, however, test cases can be non-deterministic or "flaky" \cite{luo_empirical_2014} meaning they can fail intermittently without any modification to the code being tested or the test itself. \Ac{UI} tests, which frequently require asynchronous activities and limited computer resources, might be especially susceptible to unstable testing \cite{romano_empirical_2021}.

Flaky tests can waste resources, increase development time and erode confidence in the testing procedure. Google estimates that about 16\% of tests are flaky \cite{micco_state_2017}. Even given the importance of testing, the detection of flaky tests is still a challenge. The most common approach to identify flaky tests is rerunning \cite{lam_idflakies_2019}. However, rerunning has several downsides, such as consuming significant computational resources, generating false positives, and potentially missing certain types of flaky tests \cite{bell_deflaker_2018, luo_empirical_2014}.

To provide an alternative for \ac{E2E} tests, his thesis focuses on identifying flaky failures by applying a technique for detecting flaky unit test failures proposed by Bell et al in their DeFlaker paper \cite{bell_deflaker_2018} to \ac{E2E} tests. In DeFlaker Bell et al show that by comparing the code covered by running a test with the changes since the last passing of that test, a good estimation can be made to wether or not a failure is flaky. To evaluate our approach we selected two open source projects as case studies. Artemis \cite{krusche_artemis_2018} is a web-based learning platform for programming exercises. It uses Java for server code and JavaScript for the clientside code. n8n \cite{noauthor_n8n_2023} is a workflow automation tool written in TypeScript for both server and client. Both projects use Cypress \cite{noauthor_cypress-iocypress_2023} for \ac{E2E} testing.

The primary objectives of this thesis are as follows:
\begin{enumerate}
	\item Develop a tool to collect code coverage information for \ac{E2E} across server and client code.
	\item Determine if a failure is likely to be flaky based on the changes since the last pass and the covered code.
	\item Evaluate recall and precision of the approach compared to rerunning the test.
\end{enumerate}

The remainder of this thesis is organized as follows:
In \cref{background} we will give some background information on the topic of \ac{E2E} testing, flaky tests, and coverage collection. 
We will review current alternative approaches in \cref{related_work}.
In \cref{methodology} we will describe the methodology used in the project, including the instrumentation, data collection and environment setup.
In \cref{evaluation} we will present the results of the project and answer our research questions. 
In \cref{threats} we will discuss the limitations of the project and in \cref{conclusion} we will summarize results of the evaluation and discuss future work.



%------- chapter 2 -------

\chapter{Background}\label[chapter]{background}

\section{E2E Testing}

\section{Flaky Tests}

\section{Coverage Collection}



%------- chapter 3 -------

\chapter{Related Work}\label[chapter]{related_work}

\textit{Note: Describe related work regarding your topic and emphasize your (scientific) contribution in \textbf{contrast} to existing approaches / concepts / workflows. Related work is usually current research by others and you defend yourself against the statement: ``Why is your thesis relevant? The problem was already solved by XYZ.'' If you have multiple related works, use subsections to separate them.}

\section{E2E Testing}

\section{Flaky Tests}

\section{Coverage Collection}


%------- chapter 4 -------

\chapter{Methodology}\label[chapter]{methodology}

\section{Instrumentation}

\subsection{Server}

\subsection{Client}

\section{Change Collection}

\subsection{File Level}

\subsection{Line Level}

\section{Implementation}

%------- chapter 5 -------

\chapter{Evaluation}\label[chapter]{evaluation}

\section{Research Questions}

\section{Evaluation Setup}

\subsection{Artemis (Bamboo)}

\subsection{n8n (GitHub)}

\section{Evaluation Results}

\section{Discussion}


%------- chapter 6 -------

\chapter{Threats to Validity}\label[chapter]{threats}

\section{Threats to Internal Validity}

\section{Threats to External Validity}


%------- chapter 7 -------

\chapter{Conclusion}\label[chapter]{conclusion}

\textit{Note: This chapter includes the status of your thesis, a conclusion and an outlook about future work.}

\section{Status}

\textit{Note: Describe honestly the achieved goals (e.g. the well implemented and tested use cases) and the open goals here. if you only have achieved goals, you did something wrong in your analysis.}

\begin{itemize}
	\item [\Circle]
	\item [\LEFTcircle]
	\item [\CIRCLE]
\end{itemize}

\subsection{Realized Goals}

\textit{Note: Summarize the achieved goals by repeating the realized requirements or use cases stating how you realized them.}

\subsection{Open Goals}

\textit{Note: Summarize the open goals by repeating the open requirements or use cases and explaining why you were not able to achieve them. \textbf{Important:} It might be suspicious, if you do not have open goals. This usually indicates that you did not thoroughly analyze your problems.}

\section{Conclusion}

\textit{Note: Recap shortly which problem you solved in your thesis and discuss your \textbf{contributions} here.}

\section{Future Work}

\textit{Note: Tell us the next steps  (that you would do if you have more time). Be creative, visionary and open-minded here.}



\appendix

\chapter{e.g. Questionnaire}

\textit{Note: If you have large models, additional evaluation data like questionnaires or non summarized results, put them into the appendix.}


\clearpage

\listoffigures
\clearpage

\listoftables
\clearpage

\bibliography{thesis}
\bibliographystyle{acm}

\end{document}
