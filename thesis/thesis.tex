% Add common preamble to the document
% !TeX root = ..\thesis.tex
% \documentclass[a4paper,12pt,twoside,listof=totoc,bibliography=totoc]{report}
\documentclass[headsepline,footsepline,footinclude=false,oneside,fontsize=11pt,paper=a4,listof=totoc,bibliography=totoc,parskip=half]{scrbook}

\usepackage[utf8]{inputenc}
\usepackage[T1]{fontenc}
\usepackage{cascadia-code}
% \usepackage[scaled]{helvet}
\usepackage[sc]{mathpazo}
% \usepackage{newcent}
\usepackage{url}
% \usepackage{cite}
\usepackage[ngerman,american]{babel}
\usepackage[autostyle]{csquotes}
\usepackage[%
  backend=biber,
  url=true,
  style=numeric-comp,
  maxnames=4,
  minnames=3,
  maxbibnames=99,
  % giveninits,
  uniquename=init]{biblatex}
\usepackage{scrhack} % necessary for listings package
\usepackage{listings}
\usepackage{lstautogobble}
\usepackage[pdftex]{graphicx}
% \usepackage[hang,small,bf]{caption}
\usepackage{styles/tum}
\usepackage{tikz}
\usepackage{pgfplots}
\usepackage{pgfplotstable}
\usepackage{amsmath}
\usepackage[USenglish]{datetime2}
\usepackage{setspace}
\usepackage[l3]{csvsimple}
% \usepackage[german,english]{babel}
\usepackage{float}
\usepackage{floatflt}
% \usepackage{fancyhdr}
\usepackage{siunitx}
\usepackage{color}
\usepackage{booktabs}
\usepackage{threeparttable}
\usepackage{subcaption}
\usepackage[final]{microtype}
\usepackage{caption}
\usepackage[pdftex,bookmarks=true,plainpages=false,pdfpagelabels=true,hidelinks]{hyperref}	%TODO make yourself familiar with \label, \ref and \hyperref for referencing figures, tables, chapters, etc.
\usepackage{mdwlist}
\usepackage{enumerate}
\usepackage{array}
\usepackage{longtable}
\usepackage[utf8]{inputenc}
\usepackage[capitalize, noabbrev]{cleveref}
\usepackage{wasysym}
\usepackage{adjustbox}
\usepackage{mdframed}[framemethod=TikZ]
\usepackage[backgroundcolor=TUMLightGray!20,linecolor=TUMAccentOrange,bordercolor=TUMAccentOrange,colorinlistoftodos,tickmarkheight=0.1cm]{todonotes}

% Path for graphics
\graphicspath{{figures/}}

% Include the Thesis metadata like title, author, etc. 
\input{metadata}

\DeclareFieldFormat{urldate}{}

\AtEveryBibitem{
  \clearlist{address}
  \clearfield{date}
  \clearfield{eprint}
  %  \clearfield{doi}
  \clearfield{isbn}
  \clearfield{issn}
  \clearlist{location}
  \clearfield{month}
  \clearfield{labelmonth}
  \clearfield{series}

  \ifentrytype{book}{
    \clearfield{url}
  }
  \ifentrytype{misc}{
  }{
    \clearfield{url}
    \clearlist{publisher}
    \clearname{editor}
  }
}

\bibliography{thesis}

\setkomafont{disposition}{\normalfont\bfseries} % use serif font for headings
\linespread{1.05} % adjust line spread for mathpazo font

\sisetup{round-mode = places, round-precision = 2}
\newcommand{\fixedtwo}{\sisetup{round-mode = places, round-precision = 2}}

% Add table of contents to PDF bookmarks
\BeforeTOCHead[toc]{{\cleardoublepage\pdfbookmark[0]{\contentsname}{toc}}}

% Define TUM corporate design colors
% Taken from http://portal.mytum.de/corporatedesign/index_print/vorlagen/index_farben
\definecolor{TUMBlue}{HTML}{0065BD}
\definecolor{TUMSecondaryBlue}{HTML}{005293}
\definecolor{TUMSecondaryBlue2}{HTML}{003359}
\definecolor{TUMBlack}{HTML}{000000}
\definecolor{TUMWhite}{HTML}{FFFFFF}
\definecolor{TUMDarkGray}{HTML}{333333}
\definecolor{TUMGray}{HTML}{808080}
\definecolor{TUMLightGray}{HTML}{CCCCC6}
\definecolor{TUMAccentGray}{HTML}{DAD7CB}
\definecolor{TUMAccentOrange}{HTML}{E37222}
\definecolor{TUMAccentGreen}{HTML}{A2AD00}
\definecolor{TUMAccentLightBlue}{HTML}{98C6EA}
\definecolor{TUMAccentBlue}{HTML}{64A0C8}

\addto\extrasamerican{
  \def\lstnumberautorefname{Line}
  \def\chapterautorefname{Chapter}
  \def\sectionautorefname{Section}
  \def\subsectionautorefname{Subsection}
  \def\subsubsectionautorefname{Subsubsection}
}

\addto\extrasngerman{
  \def\lstnumberautorefname{Zeile}
}

\mdfsetup{linewidth=1pt, linecolor=TUMDarkGray, nobreak}
\newtheorem{definition}{Definition}
\usetikzlibrary{shapes,fit,positioning,arrows, calc}

% Settings for pgfplots
\pgfplotsset{compat=newest}
\pgfplotsset{
  % For available color names, see http://www.latextemplates.com/svgnames-colors
  cycle list={TUMBlue\\TUMAccentOrange\\TUMAccentGreen\\TUMSecondaryBlue2\\TUMDarkGray\\},
}
\usepgfplotslibrary{statistics}

%%%%%%%%%%%%%%%%%%%%%%%%%%%%%%%%%%%%%%%%%%%%%%%%%%%%%%%%%%%%%%%%%%%%%%%%%%%%%%%%%%%%%%%%%%%%%%%%%
% Custom config for code listings
%%%%%%%%%%%%%%%%%%%%%%%%%%%%%%%%%%%%%%%%%%%%%%%%%%%%%%%%%%%%%%%%%%%%%%%%%%%%%%%%%%%%%%%%%%%%%%%%%
\lstset{%
  numbers=left,
  showstringspaces=false,
  basicstyle=\ttfamily,
  columns=fullflexible,
  autogobble,
  keywordstyle=\bfseries\color{TUMBlue},
  stringstyle=\color{TUMAccentBlue},
  commentstyle=\color{TUMAccentGreen},
  ndkeywordstyle=\color{TUMDarkGray}\bfseries,
  captionpos=b,
  breaklines=true,
  postbreak=\mbox{\textcolor{TUMAccentGray}{$\hookrightarrow$}\space}
}

% !TeX root = ..\thesis.tex
% http://tex.stackexchange.com/questions/152829/how-can-i-highlight-yaml-code-in-a-pretty-way-with-listings
\newcommand\YAMLcolonstyle{\color{TUMDarkGray}\mdseries}
\newcommand\YAMLkeystyle{\color{black}\bfseries}
\newcommand\YAMLvaluestyle{\color{TUMAccentBlue}\mdseries}

\makeatletter

% here is a macro expanding to the name of the language
% (handy if you decide to change it further down the road)
\newcommand\language@yaml{yaml}

\expandafter\expandafter\expandafter\lstdefinelanguage
\expandafter{\language@yaml}
{
  keywords={true,false,null,y,n},
  sensitive=false,
  comment=[l]{\#},
  morecomment=[s]{/*}{*/},
  moredelim=[l][\color{orange}]{\&},
  moredelim=[l][\color{magenta}]{*},
  moredelim=**[il][\YAMLcolonstyle{:}\YAMLvaluestyle]{:},   % switch to value style at :
  morestring=[b]',
  morestring=[b]",
  literate =    {---}{{\ProcessThreeDashes}}3
                {>}{{\textcolor{red}\textgreater}}1     
                % {|}{{\textcolor{red}\textbar}}1 
                {\ -\ }{{\mdseries\ -\ }}3,
}

% switch to key style at EOL
\lst@AddToHook{EveryLine}{\ifx\lst@language\language@yaml\YAMLkeystyle\fi}
\makeatother

\newcommand\ProcessThreeDashes{\llap{\color{cyan}\mdseries-{-}-}}

%%%%%%%%%%%%%%%%%%%%%%%%%%%%%%%%%%%%%%%%%%%%%%%%%%%%%%%%%%%%%%%%%%%%%%%%%%%%%%%%%%%%%%%%%%%%%%%%%
% Custom Commands for this template
%%%%%%%%%%%%%%%%%%%%%%%%%%%%%%%%%%%%%%%%%%%%%%%%%%%%%%%%%%%%%%%%%%%%%%%%%%%%%%%%%%%%%%%%%%%%%%%%%

% Annotate feedback you received 
\newcommand{\feedback}[1]{\todo[inline,bordercolor=TUMAccentGreen]{#1}}
\newcommand{\comment}[1]{\todo[bordercolor=TUMAccentLightBlue,linecolor=TUMAccentLightBlue]{#1}}

% State what is missing in this spot
\newcommand{\missing}[1]{\todo[inline,color=yellow,bordercolor=TUMGray,linecolor=TUMGray]{#1}}

% Inline to do note: 
\newcommand{\TODO}[1]{\todo[inline]{#1}}

\newcommand{\addref}{\todo[bordercolor=red,linecolor=red]{Add reference.}}
\newcommand{\rewrite}[1]{\todo[bordercolor=TUMAccentBlue,linecolor=TUMAccentBlue]{#1}}


%%%%%%%%%%%%%%%%%%%%%%%%%%%%%%%%%%%%%%%%%%%%%%%%%%%%%%%%%%%
% Theses specific packages go here
%%%%%%%%%%%%%%%%%%%%%%%%%%%%%%%%%%%%%%%%%%%%%%%%%%%%%%%%%%%
\usepackage[printonlyused]{acronym}


%%%%%%%%%%%%%%%%%%%%%%%%%%%%%%%%%%%%%%%%%%%%%%%%%%%%%%%%%%%
% Begin of document
%%%%%%%%%%%%%%%%%%%%%%%%%%%%%%%%%%%%%%%%%%%%%%%%%%%%%%%%%%%

\begin{document}
\setlength{\evensidemargin}{22pt}
\setlength{\oddsidemargin}{22pt}


\hypersetup{pdfborder={0 0 0}, pdfauthor={\author}, pdftitle={\title}}

\lstset{showspaces=false, numbers=left, frame=single, basicstyle=\small}

\pagenumbering{alph}

%------- Cover and Title setup -------
\begin{titlepage}
    % HACK for two-sided documents: ignore binding correction for cover page.
    % Adapted from Markus Kohm's KOMA-Script titlepage=firstiscover handling.
    % See http://mirrors.ctan.org/macros/latex/contrib/koma-script/scrkernel-title.dtx,
    % \maketitle macro.
    \oddsidemargin=\evensidemargin\relax
    \textwidth=\dimexpr\paperwidth-2\evensidemargin-2in\relax
    \hsize=\textwidth\relax

    \centering
    \oTUM{4cm}

    \vspace{5mm}
    {\huge\MakeUppercase{School of Computation, Information and Technology \\ -- Informatics --} \par }

    \vspace{5mm}
    {\large\MakeUppercase{Technical University of Munich}\par}

    \vspace{15mm}
    {\Large {\degree}'s Thesis in \program \par}

    \vspace{10mm}
    {\huge\bfseries \title \par}

    \vspace{10mm}
    {\LARGE \author}


\end{titlepage}

\begin{titlepage}
        \centering

        \oTUM{4cm}

        \vspace{5mm}
        {\huge\MakeUppercase{School of Computation, Information and Technology \\ -- Informatics --} \par }

        \vspace{5mm}
        {\large\MakeUppercase{Technical University of Munich}}

        \vspace{10mm}
        {\Large {\degree}'s Thesis in \program }

        \vspace{10mm}
        {\LARGE\bfseries \title \par}

        \vspace{5mm}
        {\LARGE\bfseries \foreignlanguage{ngerman}{\titleGer} \par}

        \vspace{10mm}
        \begin{tabular}{l l}
                Author:          & \author         \\
                Supervisor:      & \supervisor     \\
                Advisor:         & \advisor        \\
                Advisor:         & \advisorTwo        \\
                Submission Date: & \date \\
        \end{tabular}
\end{titlepage}


%------- Disclaimer -------
\newpage
\thispagestyle{empty}
\mbox{}
\clearpage
\thispagestyle{empty}
\vspace*{0.8\textheight}
\noindent
I confirm that this \MakeLowercase{\degree}'s thesis is my own work and I have documented all sources and material used.

\vspace{15mm}
\noindent
Munich, \date \hspace{\stretch{1}} \author
\newpage



%------- Acknowledgements -------
\newpage
\thispagestyle{empty}
\mbox{}
\chapter*{Acknowledgements}




\pagenumbering{roman}

%------- Abstracts -------
\selectlanguage{english}
\addchap{Abstract}

Software testing plays a critical role in ensuring system quality, reliability, and performance. 
However, the effectiveness of testing is limited by the presence of flaky tests, which are tests that can pass or fail randomly with the same version of the \acf*{sut} and configuration.
\acf*{e2e} tests in particular can be susceptible to flaky behavior, posing significant challenges to the software development process. 
While existing research has primarily focused on the detection of flaky unit tests, this thesis addresses the detection of flaky failures in \acs*{e2e} tests. 
We adapt a technique from \citeauthor*{bell_deflaker_2018} for detecting flaky Java unit test failures and apply it to \acs*{e2e} tests, which involve multiple programming languages and a different context for test execution. Our approach requires the instrumentation of code to collect coverage data during test execution. 
To evaluate our methodology, we selected two open source projects, \textsc{ArTEMiS} and \textsc{n8n}, which use the Cypress framework for \acs*{e2e} testing.

Our contributions include a novel methodology for detecting flaky failures in \acs*{e2e} testing, an evaluation of our approach on two open source projects, and guidelines for practitioners. 
Our evaluation shows that instrumentation can have a significant impact on test execution, and that our approach correctly identified flaky failures 42\% of the time. 
73\% of the failures identified by our approach were indeed flaky, and 27\% were false positives.
The results indicate that while the adapted approach has potential, it is not directly applicable to \acs*{e2e} tests, and further research is needed to improve its performance. 
This is due to the high implementation complexity of the approach, the impact of instrumentation on test execution, and the low recall of the approach.
Future work should also explore the applicability of our approach to different technology stacks, extend the coverage collection tool to support other test frameworks, and investigate the impact of instrumentation on test execution.
\clearpage
\selectlanguage{german}
\chapter{Abstract}

Software-Tests sind ein entscheidender Aspekt der Softwareentwicklung, um die Qualität und Zuverlässigkeit von Systemen sicherzustellen. Obwohl Testfälle eigentlich deterministisch sein sollten, können sogenannte "flaky" Tests nicht-deterministisches Verhalten aufweisen, was zu intermittierenden Fehlern führt, ohne dass Änderungen am zu testenden System (SUT) oder am Test selbst vorgenommen wurden. Diese Arbeit zielt darauf ab, flaky Fehler in End-to-End (E2E)-Tests zu erkennen, indem die von Bell et al. in "DeFlaker" vorgeschlagene Methode zum Erkennen von flaky Fehlern in Unit-Tests angewendet wird. Durch die Konzentration auf flaky Fehler anstelle von flaky Tests soll unser Ansatz während der Softwareentwicklung im Continuous Integration (CI)-Prozess einen höheren Mehrwert bieten.

Wir schlagen eine Methodik zur Erkennung von flaky Fehlern in E2E-Tests vor und bewerten deren Leistung anhand von zwei Open-Source-Projekten als Fallstudien: Artemis, eine webbasierte Lernplattform für Programmierübungen, und n8n, ein Workflow-Automatisierungstool. Beide Projekte verwenden Cypress für E2E-Tests. Unsere Bewertung beurteilt die Auswirkungen unserer Methodik auf die Testausführung und ihre Fähigkeit, flaky Fehler korrekt zu identifizieren. Unser Ansatz funktioniert gut und identifiziert flaky Fehler in 90\% der Fälle korrekt, wobei es in drei Fällen falsche Positive gab, bei denen unser Ansatz einen Fehler als flaky einstufte, der es jedoch nicht war.

Diese Arbeit bietet einen Überblick über E2E-Tests, flaky Tests, Erfassung von Codeabdeckung und alternative Ansätze zur Erkennung von flaky Tests. Sie beschreibt auch die Methodik, Fallstudien, Ergebnisse der Bewertung, Einschränkungen und zukünftige Arbeiten zur Erkennung von flaky Fehlern in E2E-Tests.

\clearpage
\selectlanguage{english}

%------- Table of contents -------
\tableofcontents
\clearpage

\clearpage

%------- Common Acronyms -------
% !TeX root = ..\thesis.tex

\addchap{Acronyms}
\begin{acronym}
    \itemsep-.25\baselineskip
    \acro{api}[API]{Application Programming Interface}
    \acro{ci}[CI]{Continuous Integration}
    \acro{cli}[CLI]{Command Line Interface}
    \acro{dom}[DOM]{Document Object Model}
    \acro{e2e}[E2E]{End-to-End}
    \acro{fn}[FN]{False Negative}
    \acro{fp}[FP]{False Positive}
    \acro{gui}[GUI]{Graphical User Interface}
    \acro{mcc}[MCC]{Matthews Correlation Coefficient}
    \acro{pr}[PR]{Pull Request}
    \acro{rq}[RQ]{Research Question}
    \acro{sut}[SUT]{System Under Test}
    \acro{tn}[TN]{True Negative}
    \acro{tp}[TP]{True Positive}
    \acro{tum}[TUM]{Technical University of Munich}
    \acro{ui}[UI]{User Interface}
    \acro{od}[OD]{order-dependent}
    \acro{nod}[NOD]{non-order-dependent}
\end{acronym}

\pagenumbering{arabic}

\fancyhead{}
\pagestyle{fancy}
\fancyhead[LE]{\slshape \leftmark}
\fancyhead[RO]{\slshape \rightmark}
\headheight=15pt




%------- chapter 1 -------

\chapter{Introduction}\label{chapter:introduction}

\begin{itemize}
	\item \textbf{Motivation}: Discuss the importance of testing in software development and the challenges faced due to flaky tests, particularly in end-to-end tests and UI tests.

	\item \textbf{Problem Statement}: Define the problem of flaky tests and how they impact software development processes, focusing on the issues faced in projects like Artemis and n8n, which use Java and JavaScript.

	\item \textbf{Research Objectives}: Introduce the goals of the thesis, including the development of a tool to collect code coverage information for end-to-end tests and the detection of flaky test failures based on collected information and code changes.

	\item \textbf{Methodology Overview}: Provide a brief overview of the methodology used in the project, including the instrumentation of the n8n project, data collection, and evaluation process.

	\item \textbf{Significance}: Explain the potential benefits of the research, such as improving the efficiency of software development, reducing the time spent on addressing flaky tests, and providing insights into the causes of flaky test failures.

	\item \textbf{Thesis Structure}: Outline the organization of the thesis, briefly describing the contents of each chapter.
\end{itemize}

\textit{Note: Introduce the topic of your thesis, e.g. with a little historical overview.}

\section{Problem}

\textit{Note: Describe the problem that you like to address in your thesis to show the importance of your work. Focus on the negative symptoms of the currently available solution.}

\section{Motivation}

\textit{Note: Motivate scientifically why solving this problem is necessary. What kind of benefits do we have by solving the problem?}

\section{Objectives}

\textit{Note: Describe the research goals and/or research questions and how you address them by summarizing what you want to achieve in your thesis, e.g. developing a system and then evaluating it.}

\section{Outline}

\textit{Note: Describe the outline of your thesis}




%------- chapter 2 -------

\chapter{Background}

\section{E2E Testing}

\section{Flaky Tests}

\section{Coverage Collection}



%------- chapter 3 -------

\chapter{Related Work}

\textit{Note: Describe related work regarding your topic and emphasize your (scientific) contribution in \textbf{contrast} to existing approaches / concepts / workflows. Related work is usually current research by others and you defend yourself against the statement: ``Why is your thesis relevant? The problem was already solved by XYZ.'' If you have multiple related works, use subsections to separate them.}




%------- chapter 4 -------

\chapter{Requirements Analysis}

\textit{Note: This chapter follows the Requirements Analysis Document Template in \cite{bruegge2004object}.
	\textbf{Important:} Make sure that the whole chapter is independent of the chosen technology and development platform. The idea is that you illustrate concepts, taxonomies and relationships of the application domain independent of the solution domain!
	Cite \cite{bruegge2004object} several times in this chapter.}

\section{Overview}

\textit{Note: Provide a short overview about the purpose, scope, objectives and success criteria of the system that you like to develop.}

\section{Current System}

\textit{Note: This section is only required if the proposed system (i.e. the system that you develop in the thesis) should replace an existing system.}

\section{Proposed System}

\textit{Note: If you leave out the section ``Current system'', you can rename this section into ``Requirements''.}

\subsection{Functional Requirements}

\textit{Note: List and describe all functional requirements of your system. Also mention requirements that you were not able to realize. The short title should be in the form ``verb objective''}

\begin{itemize}
	\item [FR1] \textbf{Short Title}: Short Description.
	\item [FR2] \textbf{Short Title}: Short Description.
	\item [FR3] \textbf{Short Title}: Short Description.
\end{itemize}

\subsection{Nonfunctional Requirements}

\textit{Note: List and describe all nonfunctional requirements of your system. Also mention requirements that you were not able to realize. Categorize them using the FURPS+ model described in \cite{bruegge2004object} without the category \textbf{functionality} that was already covered with the functional requirements.}

\begin{itemize}
	\item [NFR1] \textbf{Category}: Short Description.
	\item [NFR2] \textbf{Category}: Short Description.
	\item [NFR3] \textbf{Category}: Short Description.
\end{itemize}

\section{System Models}

\textit{Note: This section includes important system models for the requirements analysis.}

\subsection{Scenarios}

\textit{Note: If you do not distinguish between visionary and demo scenarios, you can remove the two subsubsections below and list all scenarios here.}

\subsubsection{Visionary Scenarios}

\textit{Note: Describe 1-2 visionary scenario here, i.e. a scenario that would perfectly solve your problem, even if it might not be realizable. Use free text description.}

\subsubsection{Demo Scenarios}

\textit{Note: Describe 1-2 demo scenario here, i.e. a scenario that you can implement and demonstrate until the end of your thesis. Use free text description.}

\subsection{Use Case Model}

\textit{Note: This subsection should contain a UML Use Case Diagram including roles and their use cases. You can use colors to indicate priorities. Think about splitting the diagram into multiple ones if you have more than 10 use cases.
	\textbf{Important:} Make sure to describe the most important use cases using the use case table template (./tex/use-case-table.tex). Also describe the rationale of the use case model, i.e. why you modeled it like you show it in the diagram.}

\subsection{Analysis Object Model}

\textit{Note: This subsection should contain a UML Class Diagram showing the most important objects, attributes, methods and relations of your application domain including taxonomies using specification inheritance (see \cite{bruegge2004object}). Do not insert objects, attributes or methods of the solution domain.
	\textbf{Important:} Make sure to describe the analysis object model thoroughly in the text so that readers are able to understand the diagram. Also write about the rationale how and why you modeled the concepts like this.}

\subsection{Dynamic Model}

\textit{Note: This subsection should contain dynamic UML diagrams. These can be a UML state diagrams, UML communication diagrams or UML activity diagrams.\textbf{Important:} Make sure to describe the diagram and its rationale in the text. \textbf{Do not use UML sequence diagrams.}}

\subsection{User Interface}

\textit{Note: Show mockups of the user interface of the software you develop and their connections / transitions. You can also create a storyboard. \textbf{Important:} Describe the mockups and their rationale in the text.}



%------- chapter 5 -------

\chapter{System Design}

\textit{Note: This chapter follows the System Design Document Template in \cite{bruegge2004object}.
	You describe in this chapter how you map the concepts of the application domain to the solution domain. Some sections are optional, if they do not apply to your problem.
	Cite \cite{bruegge2004object} several times in this chapter.}

\section{Overview}

\textit{Note: Provide a brief overview of the software architecture and references to other chapters (e.g. requirements analysis), references to existing systems, constraints impacting the software architecture.}

\section{Design Goals}

\textit{Note: Derive design goals from your nonfunctional requirements, prioritize them (as they might conflict with each other) and describe the rationale of your prioritization. Any trade-offs between design goals (e.g., build vs. buy, memory space vs. response time),
	and the rationale behind the specific solution should be described in this section}

\section{Subsystem Decomposition}

\textit{Note: Describe the architecture of your system by decomposing it into subsystems and the services provided by each subsystem. Use UML class diagrams including packages / components for each subsystem.}

\section{Hardware Software Mapping}

\textit{Note: This section describes how the subsystems are mapped onto existing hardware and software components. The description is accompanied by a UML deployment diagram. The existing components are often off-the-shelf components. If the components are distributed on different nodes, the network infrastructure and the protocols are also described.}

\section{Persistent Data Management}

\textit{Note: Optional section that describes how data is saved over the lifetime of the system and which data. Usually this is either done by saving data in structured files or in databases. If this is applicable for the thesis, describe the approach for persisting data here and show a UML class diagram how the entity objects are mapped to persistent storage.
	It contains a rationale of the selected storage scheme, file system or database, a description of the selected database and database administration issues.}

\section{Access Control}

\textit{Note: Optional section describing the access control and security issues based on the nonfunctional requirements in the requirements analysis. It also describes the implementation of the access matrix based on capabilities or access control lists, the selection of  authentication mechanisms and the use of encryption algorithms.}

\section{Global Software Control}

\textit{Note: Optional section describing the control flow of the system, in particular, whether a monolithic, event-driven control flow or concurrent processes have been selected, how requests are initiated and specific synchronization issues}


\section{Boundary Conditions}

\textit{Note: Optional section describing the use cases how to start up the separate components of the system, how to shut them down, and what to do if a component or the system fails.}





%------- chapter 6 -------

\chapter{Case Study / Evaluation}

\textit{Note: If you did an evaluation / case study, describe it here.}

\section{Design}

\textit{Note: Describe the design / methodology of the evaluation and why you did it like that. E.g. what kind of evaluation have you done (e.g. questionnaire, personal interviews, simulation, quantitative analysis of metrics, what kind of participants, what kind of questions, what was the procedure?}

\section{Objectives}

\textit{Note: Derive concrete objectives / hypotheses for this evaluation from the general ones in the introduction.}

\section{Results}

\textit{Note: Summarize the most interesting results of your evaluation (without interpretation). Additional results can be put into the appendix.}

\section{Findings}

\textit{Note: Interpret the results and conclude interesting findings}

\section{Discussion}

\textit{Note: Discuss the findings in more detail and also review possible disadvantages that you found}

\section{Limitations}

\textit{Note: Describe limitations and threats to validity of your evaluation, e.g. reliability, generalizability, selection bias, researcher bias}



%------- chapter 7 -------

\chapter{Summary}

\textit{Note: This chapter includes the status of your thesis, a conclusion and an outlook about future work.}

\section{Status}

\textit{Note: Describe honestly the achieved goals (e.g. the well implemented and tested use cases) and the open goals here. if you only have achieved goals, you did something wrong in your analysis.}

\begin{itemize}
	\item [\Circle]
	\item [\LEFTcircle]
	\item [\CIRCLE]
\end{itemize}

\subsection{Realized Goals}

\textit{Note: Summarize the achieved goals by repeating the realized requirements or use cases stating how you realized them.}

\subsection{Open Goals}

\textit{Note: Summarize the open goals by repeating the open requirements or use cases and explaining why you were not able to achieve them. \textbf{Important:} It might be suspicious, if you do not have open goals. This usually indicates that you did not thoroughly analyze your problems.}

\section{Conclusion}

\textit{Note: Recap shortly which problem you solved in your thesis and discuss your \textbf{contributions} here.}

\section{Future Work}

\textit{Note: Tell us the next steps  (that you would do if you have more time). Be creative, visionary and open-minded here.}



\appendix

\chapter{e.g. Questionnaire}

\textit{Note: If you have large models, additional evaluation data like questionnaires or non summarized results, put them into the appendix.}


\clearpage

\listoffigures
\clearpage

\listoftables
\clearpage

\bibliography{thesis}
\bibliographystyle{alpha}

\end{document}
